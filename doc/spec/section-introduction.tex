
\chapter{Introduction}



\section{Definition of terms}

The YARP library supports transmission of a stream of
user data across
various protocols -- TCP, UDP, MCAST (multi-cast), shared memory, QNX
message passing -- insulating a user of the library from the
idiosyncratic details of the network technology used.  We call these
these low-level protocols the ``Carriers'', to distinguish them from
the higher-level protocols we will be concerned with here.

For the purposes of YARP, communication takes place through
``Connections'' between named entities called ``Ports''.
These form a directed graph, the ``YARP Network'', where Ports are the nodes,
and Connections are the edges.

Each Port is assigned a unique name, such as ``/motor/wheels/left''.  
Every Port is registered by name with
a ``name server''.  The goal is to ensure that if you know the name
of a Port, that is all you need in order to be able to 
communicate with it from any machine.

The purpose of Ports is to move ``Content'' (sequences of bytes representing 
user data) from
one thread to another (or several others) across process and machine
boundaries.  The flow of data can be manipulated and monitored
externally (e.g. from the command-line) at run-time.  In other words,
the edges in the YARP Network are entirely mutable.

Ports are specialized into InputPorts and OutputPorts.
Any OutputPort can send Content to any number of InputPorts.  Any
InputPort can receive Content from any number of OutputPorts.  If an
OutputPort is configured to send Content to an InputPort, they are
said to have a Connection.  Connections can be freely added or 
removed, and may use different Carriers.

The YARP name server is a server that tracks information about ports.
It indexes this information by name, playing a role analogous to
DNS on the internet.
%
To communicate with a port, the properties of that port need to be
known (the machine it is running on, the socket it is listening on,
the carriers it supports).  The YARP name server offers a convenient
place to store these properties, so that only the name of the port is
needed to recover them.
%
The protocol for communicating with the name server
and its operation is specified in section \ref{sect:name-protocol}.

Here are the specifications available in this document:
%
\begin{itemize} \pflist

\item Properties of a YARP network.

\item Properties of the YARP utility.

\item Properties of the YARP name server and the protocol used 
for communicating with it.

\item Properties of ports and the protocol used
for communicating with them.

\end{itemize}

\noindent
This document does not contain specification of standard libraries.



\section{Properties of a YARP network}

A YARP network consists of the following entities: a set of
ports, a set of connections, a set of names, a name server, and a set
of registrations.

\begin{itemize} \pflist

\item Every port has a unique name.

\item Every connection has a source port and a target port.

\item Each port maintains a list of all connections for which it
is the target port.

\item Each port maintains a list of all connections for which it
is the source port.

\item There is a single name server in a YARP network.

\item The name server maintains a list of registrations.  Each 
registration contains information about a single port, identified
by name.

\end{itemize}


\noindent
Communication within a YARP network can occur between two ports,
between a port and the name server, between a port and an
external entity, and between the name server and an external entity.


\begin{itemize} \pflist

\item Communication between two ports occurs if and only if there
is a connection between them.  That communication uses the
``connection protocol''.

\item Connections involving a port can be created, destroyed, or
queried by communication between an external entity and that port.
This is done by sending ``port commands'' using the YARP
connection protocol.

\item Ports communicate with the name server using the 
``YARP name server protocol''.  Such communication is needed
to create, remove, and query registrations.

\item External entities can also use the YARP name server protocol
to query the name server.

\end{itemize}

\noindent
The standard YARP companion utility can be used to create a
name server, and also to act as an external entity for querying and
modifying the YARP network.
